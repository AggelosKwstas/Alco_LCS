\documentclass[a4paper,12pt]{report}
\usepackage[colorlinks]{hyperref} 

\title{
	\usefont{OT1}{bch}{b}{n}
	\normalfont \normalsize \textsc{Πανεπιστήμιο Ιωαννίνων Τμήμα Πληροφορικής και Τηλεπικοινωνίων} \\ [1em]
	\normalfont \normalsize \textsc{Αλγόριθμοι και πολυπλοκότητα} \\[0.5cm]
	\huge Το πρόβλημα της Μέγιστης κοινής υποακολουθίας ({\lat LCS = Longest Common Subsequence}).
	\\[0.5cm]
}
\author{ Περσεφόνη Πατσέα 
	\\Α.Μ.: 16106 -1731 
	\\ 7ο Εξάμηνο}

\usepackage{pdfescape}
\usepackage[top=2cm,bottom=2cm,left=3cm,right=2cm]{geometry} % Περιθώρια
\usepackage{infwarerr}
\usepackage{keyval}
\usepackage{kvoptions}
\usepackage{kvsetkeys}
\usepackage{kvdefinekeys}
\usepackage{hycolor}
\usepackage{letltxmacro}
\usepackage{auxhook}
\usepackage{intcalc}

\usepackage[utf8]{inputenc}
\usepackage[english,greek]{babel}
\usepackage{graphics}
\usepackage{graphicx}
\graphicspath{{./photos/}}

\addto\captionsgreek{%
	\renewcommand{\figurename}{Εικόνα}%
}
\usepackage{chngcntr}
\counterwithout{figure}{chapter}
\newcommand{\lat} {\latintext} 

\begin{document}
	\maketitle
	
	\section*{ΠΕΡΙΛΗΨΗ} 
	Στη συγκεκριμένη άσκηση ζητούνταν η επίλυση του προβλήματος της μέγιστης κοινής υποακολουθίας {\lat [Bla20], [Sta98]}. Η επίλυση του προβλήματος, μεταξύ άλλων, έχει εφαρμογή στη βιοπληροφορική, και ειδικότερα στην ανάλυση ομοιοτήτων σε ακολουθίες {\lat DNA}. Οι ακολουθίες {\lat DNA} αναπαρίστανται με συμβολοσειρές που σχηματίζονται από 4 χαρακτήρες ({\lat A,G,C,T}) που αναπαριστούν τα νουκλεοτίδια, αδενίνη, γουανίνη, κυτοσίνη και θυμίνη.
	\vspace{0.5cm}
	
	\section{Εισαγωγή}
	Έστω ότι δίνονται δύο συμβολοσειρές {\lat X} και {\lat Y} με μήκη {\lat m} και {\lat n} αντίστοιχα. Στο πρόβλημα της μέγιστης κοινής υποακολουθίας ζητείται να βρεθεί η μέγιστη σε μήκος υποακολουθία που εντοπίζεται και στις δύο συμβολοσειρές {\lat X} και {\lat Y}. Η υποακολουθία αποτελείται από χαρακτήρες που μπορούν να εντοπιστούν και στις δύο συμβολοσειρές με την ίδια σειρά από αριστερά προς τα δεξιά.
	\\Ο αλγόριθμος ωμής δύναμης ({\lat Brute Force}) δημιουργεί όλες τις υποακολουθίες του {\lat X} $({\lat {2}^{m}}$ σε πλήθος υποακολουθίες) και ελέγχει ποια είναι η μεγαλύτερη που υπάρχει και στο {\lat Y}.
	\vspace{0.5cm}
	
	\section{Αποτελέσματα}
	Για την συγγραφή των πειραμάτων χρησιμοποιήθηκε η {\lat Python 3.10.0} και το {\lat Visual Studio Code}. Τα χαρακτηριστικά του υπολογιστή είναι {\lat i5 7600 (3.50 GHz), 16.0 GB RAM}.  
	
	\vspace{0.7cm}
	
	\textbf{Οδηγίες Εκτέλεσης του Κώδικα:}
	\\Για να εκτελέσουμε τον κώδικα, κάνουμε τα εξής βήματα:
	\\ {\lat i)} Ανοίγουμε το {\lat cmd (Windows + R)}
	\\ {\lat ii)} Πηγαίνουμε στο φάκελο που είναι αποθηκευμένος ο κώδικας
	\\ {\lat iii)} Πατάμε την εντολή {\textit {\lat python file\_name.py}}, προκειμένου να γίνει η εκτέλεση του κώδικα
	 
	\vspace{0.3cm}
	
	\textbf {\emph {Για το {\lat main.py}}}:
	\\({\textit {\lat python -u " c:{\textbackslash Users}{\textbackslash  user}{\textbackslash Desktop}}{\textbackslash Εργασία 3}{\textbackslash Κώδικες}{\textbackslash {\lat main.py"}}})
	
	\vspace{0.3cm}
	
	\textbf {\emph {Για το {\lat UnitTest.py}}}:
	\\({\textit {\lat python -u " c:{\textbackslash Users}{\textbackslash  user}{\textbackslash Desktop}}{\textbackslash Εργασία 3}{\textbackslash Κώδικες}{\textbackslash {\lat UnitTest.py"}}})
	
	\vspace{0.7cm}
	
	
	\textbf{Αλγόριθμος {\lat Longest Common Subsequence}}:
	\\ \underline {\emph {Δήλωση Προβλήματος {\lat LCS}:}}  Με δεδομένες δύο ακολουθίες, βρίσκεται το μήκος της μεγαλύτερης υποακολουθίας που υπάρχει και στις δύο. Μία υποακολουθία είναι μία ακολουθία που εμφανίζεται με την ίδια σχετική σειρά, αλλά όχι απαραίτητα συνεχόμενη.
	\vspace{0.1cm}
	\\Για την εύρεση της πολυπλοκότητας ωμής δύναμης, πρέπει πρώτα να είναι γνωστός ο αριθμός των πιθανών διαφορετικών υποακολουθιών μιας συμβολοσειράς με μήκος {\lat n}. Δηλαδή θα πρέπει πρώτα να βρεθεί ο αριθμός των υποακολουθιών με μήκη που κυμαίνονται από: {\lat 1,2,…,n-1} .
	\vspace{0.2cm} 
	\\Από τη θεωρία μετάθεσης και του συνδυασμού, ισχύει ότι ο αριθμός των συνδυασμών με 1 στοιχείο είναι ${\lat ^{n} {C}_{1}}$. Ο αριθμός των συνδυασμών με 2 στοιχεία είναι ${\lat ^{n} {C}_{2}}$ , και ούτω καθεξής. Επομένως, για {\lat n} στοιχεία ισχύει: ${\lat ^{n} {C}_{0} + ^{n} {C}_{1} + ^{n} {C}_{2} + … + ^{n} {C}_{n}  = {2}^{n}}$. Άρα μία συμβολοσειρά μήκους n έχει ${\lat {2}^{n} - 1}$ διαφορετικές πιθανές υποακολουθίες αφού δε θεωρείται υποακολουθία με μήκος 0. Αυτό σημαίνει ότι η χρονική πολυπλοκότητα της προσέγγισης της ωμής δύναμης θα είναι ${\lat O(n * {2}^{n} )}$.                                                   
	
	
	\vspace{0.9cm}
	
	
	\textbf{Αποσπάσματα οθόνης από εκτέλεση του κώδικα:}
	
	\begin{figure}[h]
		\centering
		\includegraphics[scale=0.6]{Εικόνα_1.png}
		\caption{Έξοδος Κώδικα {\lat "main.py"}}
		\label{fig:}
	\end{figure}
	
	
	\begin{figure}[h]
		\centering
		\includegraphics[scale=0.8]{Εικόνα_2.png}
		\caption{Έξοδος Κώδικα {\lat "UnitTest.py"}}
		\label{fig:2}
	\end{figure}
	
	

	\emph {** Οι κώδικες πραγματοποιήθηκαν για 10 υποθετικές υποακολουθίες, με 20 χαρακτήρες η κάθε μία **}
	
	\vspace{0.5cm}
	
	
	\section{Συμπεράσματα}
	Τα χαρακτηριστικά των προβλημάτων στα οποία εφαρμόζεται ο δυναμικός προγραμματισμός είναι η ιδιότητα των βέλτιστων επιμέρους δομών. H αναδρομική εξίσωση περιγραφής της βέλτιστης λύσης, συνήθως, βασίζεται στην ιδιότητα των βέλτιστων επιμέρους δομών. Έτσι, η ύπαρξη της ιδιότητας των βέλτιστων επιμέρους δομών για ένα πρόβλημα αποτελεί ικανοποιητική ένδειξη ότι ο δυναμικός προγραμματισμός εφαρμόζεται σε αυτό.  
	\\Ένα άλλο χαρακτηριστικό των προβλημάτων στα οποία εφαρμόζεται ο δυναμικός προγραμματισμός, είναι ο σχετικά µμικρός αριθμός των διαφορετικών επιμέρους προβλημάτων, τα οποία εμφανίζονται κατά την εφαρμογή της μεθόδου. Με τον όρο «σχετικά μικρός», συνήθως, εννοείται ένας αριθμός επιμέρους προβλημάτων που φράσσεται άνω από κάποιο πολυώνυμο του μεγέθους της εισόδου, δια κάποια σταθερά, δηλαδή είναι ${\lat O({n}^{d})} , {\lat {d}>{0}}$.
	
\end{document}    